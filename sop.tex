\documentclass[letterpaper,12pt]{article}

\usepackage{hyperref}

\setlength{\oddsidemargin}{0in}
\setlength{\evensidemargin}{0in}
\setlength{\textwidth}{6.5in}
\setlength{\topmargin}{-.3in}
\setlength{\textheight}{9in}
\pagestyle{empty}

\begin{document}

\begin{center}
{\Large Statement of Purpose} \\[.3in]
{\large Tran Phong Binh}
\end{center}

\vspace*{.5in}

Belief is the core of humans -- it is why we live, and it is what we fight for. I believe that a great society is where diligence, justice, and gratitude are at place.  Everyone has their own belief, but only the great can make their belief come true. This essay is my proof of worth for National Taiwan University Masters program in Computer Science.

Since a very young age, I have developed strong interest and talents in math, frequently getting top three from school- to city-level math competitions. I then got a scholarship into one of the most prestigious high schools in Vietnam (comparable to Taipei Municipal Jianguo High School), and aced a national Third Prize in Informatics in 2015 (Vietnam National Olympiad in Informatics, similar to Taiwan Olympiad in Informatics, challenged students with highly arduous algorithmic problems). Fast forward five years, I now study in Taiwan, as an awardee of Ministry of Education (MOE) Taiwan Scholarship.

During my time here, I have proven myself as a gifted student in Computer Science, consistently gaining the highest GPA in my department since I was a freshman. In the course \textit{Introduction to Computer Science} lectured by Professor Chin-Yun Hsieh, I was the first to attain an absolute GPA of 100 in the last decade. In the summer of 2019, I applied for an internship at Reallusion and won a seat over all of my third- and fourth-year seniors. At the start of this year, I participated in two recruitment events by Microsoft Taiwan -- \textit{Career Hack} and \textit{Intern for a Day}. In the former, I ranked Taiwan top 100/800 in the Coding Test and was the only Taipei Tech student admitted to the final round. The latter event was my proof of worth over top Taiwanese students -- I was the first person among all teams to solve problem \textit{Calculate One} and was awarded, whilst my Computer Science teammates from National Taiwan University and National Tsing Hua University together only settled half of another problem. In May 2020, upon acknowledging Taipei Tech had no training classes for ACM ICPC, I created one. I led two other freshmen, teaching myself and them on advanced algorithms, directing 39 training seminars and ultimately, was rewarded with incredible achievements -- a Gold Medal at ACM ICPC Taiwan National Contest for Technology Universities, and an Honorable Mention at Taiwan National Collegiate Programming Contest (ranking 19.7\% nationally).

Of all fields of study in Computer Science, I am most intrigued by machine learning. The fact that machines can learn is breathtaking in its own right to any curious human being. Yet, the major reason that sparks my interest in pursuing a research career in machine learning is that it opens the door to make my belief come true -- to transform society into a place of diligence, justice, and gratitude. Using computational modeling and knowledge in neuroscience, I am confident that the root of anthropology and how humans behave can be interpreted, such understanding can in turn yield world-changing policies that I can execute using rhetoric (I have been spending 3 hours per week on studying rhetoric since last year), just like how Abraham Lincoln, Martin Luther King, and Yat-Sen Sun materialized their dreams. With this in mind, on 27th October 2020, I contacted Professor Chih-Jen Lin to start conducting research in machine learning under his guidance. In three weeks, I learnt neural network and wrote a 97\% accuracy model for recognizing handwritten digits from pure gradient descent maths (integrating input normalization, He weight initialization and Swish activation function). One week later, I derived support vector machine (SVM) for binary classification and successfully implemented the algorithm from scratch (integrating polynomial kernel). During this time, I also delved into deep mathematical concepts (because math is both fun, and so incredibly powerful that, in a way, it dictates the entire universe) and can now articulate what Fourier Transform is and how the formula can be devised from a visual perspective, how Lagrange multiplier implies its geometric intuition, and what is so beautiful about the Euler's identity, etc. I have been spending 15 hours per week researching \(\nu\)-SVM for multiclass classification -- an exciting learning opportunity assigned by Professor Lin -- and 9 hours researching machine learning techniques of my interest (the most recent one being recurrent neural network).

Among all research institutions, I want to apply for a Masters degree at National Taiwan University for two principal reasons. First and foremost, it was the National Taiwan University that Professor Lin opened his machine learning lab. He believes that deep and solid scientific investigation is necessary to solve real-world problems. I endorse his belief and want to become an official student of his and the institution he is working for. Secondly, the Machine Learning and Data Mining Group is the place I want to have all my research conducted. I had been to two labs here in Taipei Tech -- one at Computer Science and Information Engineering Department, and the other at Electrical Engineering Department, but none had proven to be as competent as this one. The lab provides me with the freedom of research, opportunities to raise my social reputation (many alumni are employed at big techs such as Google, Microsoft, and Apple), and most importantly, smart people that I can learn from and co-operate with.

My ultimate research goal is to read the brain. More specifically, I want to use computational modeling to decipher what each neuron in the brain represents and how they talk to one another. This will be a transformative contribution to the machine learning research community, as it reveals the missing piece to fully model artificial intelligence after humans. On a personal aspect, this serves my lifetime goal -- to transform society into a place of diligence, justice, and gratitude. My method is to let computers learn our brain using robust mathematical models and heuristics. That is, my research career focuses on machine learning and aims to be math-heavy, which aligns with Professor Lin's research orientation.

In regards to my graduate study plan at National Taiwan University, I plan to spend as much time learning as possible, for a good reason. While I was questioning my existence the year before, I came up with a hypothesis: ``Idea is a combination of knowledge'', or
\[ i = 2^k \]
where \(i\) stands for ``idea'', and \(k\) for ``knowledge''. This theory holds firm in the realm of mathematics: Leonhard Euler could not have discovered the Euler's identity had he not understood rotation in the complex plane; Vladimir Vapnik could not have created SVM had he not known about Lagrange multipliers; and artificial neural network could not have been invented had the human brain been unlearnt. And hence learning plays a vital part in my research scheme for the next two years. In specific, I will complete my study on machine learning technologies of the '90s before February 2021. These include SVM for multiclass classification, convolutional neural network (CNN), and long short-term memory (LSTM). Simultaneously in this period, I aim to finish a convex optimization course by Stanford University that I discovered some days ago. This course covers all the cool maths that I have been craving to study -- quadratic optimization, Lagrangian duality, Newton's method, etc. This also happens to be the research core of Professor Lin and the group -- optimization methods for deep learning, which is fantastic. By then, I will be able to research more advanced machine learning techniques and applications. To get started, I will digest six most notable papers composed by my group before school starts (one paper per month). First of all, this is a feasible target as I know the authors well and can query them on the ideas and stories behind their compositions. Secondly, since the group concentrates on optimizing machine learning routines using numerical methods, I will learn a fruitful amount of math theories evolving around machine learning, which is utterly exhilarating. In addition, this will get me accustomed to the group's research area before anyone else, putting me ahead of my peers in research navigation and accomodation. By September 2021, I will have all the necessary prerequisites to study state-of-the-art publications in the field. My strategy is to read six such papers to make my entry publication comparing the world's best machine learning methods' performance. During this time, I aim to actively attend top machine learning conferences (e.g. ACM KDD, SIAM International Conference on Data Mining, ACM CIKM, etc.), reading news and articles on machine learning (e.g. KDnugget, Papers With Code, ACM Digital Library, etc.), and re-implement work of the prints I read. My second semester will be dedicated to finishing the comparison publication and researching more papers (targeting two papers per month) to obtain more and more ideas for my Masters thesis -- an approximation method for optimizing machine learning. I will also take on an internship at Google Taiwan to put my research into solving real-world problems. After two years of Masters study, I will continue my pursuit of finding the math equation of the human brain. I plan to learn more maths and to retrieve a PhD in neuroscience. The future bears many paths and options, and some may even be unforeseeable. Nevertheless, with a strong will to learn and an unwavering enthusiasm for science (especially math), I believe I can contribute greatly to your prominent university, just like I had contributed to my high school, my college, and the community where I was.

I would be immensely thrilled should I receive an acceptance letter from your university. Should you need any further information, please do not hesitate to contact me via phone +886-909-920-985 (I am currently in Taipei), or by email \href{mailto:phongbinh2511@gmail.com}{phongbinh2511@gmail.com}.

\end{document}
