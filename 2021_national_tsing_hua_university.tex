\documentclass[12pt, letterpaper, oneside]{article}

\title{\textbf{Statement of Purpose}}
\author{Tran Phong Binh\thanks{Email: \texttt{phongbinh2511@gmail.com}}}
\date{March 11, 2021}

\begin{document}

\maketitle

Here are the values that I provide: paper and tuition. Firstly, I excel at math, as stated on my resume\footnote{The resume is submitted to your university as supporting documents.}. Secondly, I conducted academic research\footnote{\texttt{phogbinh.github.io/NTUT2020FallAlgorithmicTrading/report/report.pdf}} and hence am required no trainings. Thirdly, I am globally competitive in engineering. Last week, I passed the screening round of Google Japan software engineering role, and will take part in their online coding challenge tomorrow. I also have a technical interview with Google Taiwan next week for an internship. With the three indicators above, my aim for a graduate study is straightforward: to output papers on prestigious journals of ACM. And if playing with experimental tricks such as hyperparameters and ridiculous mathematical assumptions can grant people acceptance to AAAI (contact me for such papers), I can do that too, as all required is expertise in mathematics and statistics, and a big name as paper collaborator. The second provisional value is obvious -- tuition fee is granted to your university.

What I ask for is simple: a Master's degree and a research position under professor Che-Rung Lee's supervision. I love mathematical optimization and focus on researching optimization methods for machine learning, which is why I wish to be advised by professor Lee, as math people understand.

I apply for your university because Google Taiwan demands their candidates to return to a degree after taking part in the internship. Apparently, this is win-win deal -- your university loses nothing, yet the `potential dividend' for admitting me is tremendous. Should your institution present attractive perks and benefits, I consider continuing a PhD.

\end{document}
